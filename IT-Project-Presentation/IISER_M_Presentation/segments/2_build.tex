\begin{frame}{Data and Methodology}
   \begin{enumerate}
            \item We collected prices for the following commodities in order to approximate price changes in important categories : barley, beef, cocoa, orange, sugar (food prices), natural gas and oil (energy
prices) and gold (metal prices).
            \item We used data from World Bank. Since G20 countries, which
constitutes more than 80 percent of World GDP, we decided to use G20 inflation rate to approximate global inflation. 
            \item We used data from OECD in order to calculate global inflation rate. All
our data is from 2018-05 until 2021-08 in monthly time interval.
        \end{enumerate}
\end{frame}

\begin{frame}{Data and Methodology}
    \begin{enumerate}
            \item We transformed our data by using percentage change.
            \item Since commodity prices are in nominal
values, observing percentage changes will allow us to compare it to global inflation.
            \item Afterwards,
we plotted all commodities against inflation and compared with each other in order to see possible
relationship between them.   
        \end{enumerate}
\end{frame}